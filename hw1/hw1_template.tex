\documentclass[10pt]{article}

%%% Doc layout
\usepackage{fullpage} 
\usepackage{booktabs}       % professional-quality tables
\usepackage{microtype}      % microtypography
\usepackage{parskip}
\usepackage{times}
\usepackage{graphicx}

%% Hyperlinks always black, no weird boxes
\usepackage[hyphens]{url}
\usepackage[colorlinks=true,allcolors=black,pdfborder={0 0 0}]{hyperref}

%%% Math typesetting
\usepackage{amsmath,amssymb}

%%% Write out problem statements in blue, solutions in black
\usepackage{xcolor}
\newcommand{\officialdirections}[1]{{\color{purple} #1}}

%%% Avoid automatic section numbers (we'll provide our own)
\setcounter{secnumdepth}{0}

%% --------------
%% Header
%% --------------
\usepackage{fancyhdr}
\fancyhf{}
\setlength{\headheight}{15pt}
\fancyhead[C]{\ifnum\value{page}=1 Tufts CS 145 BDL - 2026s - HW1 Submission \else \fi}
\fancyfoot[C]{\thepage} % page number
\renewcommand\headrulewidth{0pt}
\pagestyle{fancy}

%% --------------
%% Begin Document
%% --------------
\begin{document}

~\\ %% add vertical space
{\Large{\bf Student Name: TODO}}

~\\ %% add vertical space
{\bf Collaboration Statement:}

Total hours spent: TODO

I consulted the following human, textbook, or AI resources:
\begin{itemize}
\item TODO
\item TODO
\item $\ldots$	
\end{itemize}

By turning this document in, I attest that I have followed the 
\href{https://www.cs.tufts.edu/cs/145/2026s/index.html#collaboration}{[CS 145 Collaboration Policy]}. All solutions represent my own work. No solution text in this document was directly provided by other humans or artificial agents. 

\tableofcontents

\newpage

\officialdirections{
\subsection*{1a: Problem Statement}
Figure showing samples of $f$ from the prior with SE kernel at various $L$ values.
}

\subsection{1a: Solution}

\renewcommand{\figurename}{Fig.}
\renewcommand{\thefigure}{1a}
 \begin{figure}[!h]
     \centering
     \includegraphics[width=0.9\textwidth]{example-image-a.pdf}
     \label{fig:1a}
\caption{TODO YOUR BRIEF CAPTION HERE. 
}%endcaption
 \end{figure}


\officialdirections{
\subsection*{1b: Problem Statement}
Provide a short answer (1-3 complete sentences). How does the numerical value of the $L$ hyperparameter of the SE kernel control the qualitative behavior of the sampled function values?
}

\subsection{1b: Solution}
TODO YOUR SOLUTION HERE

\newpage
\officialdirections{
\subsection*{1c: Problem Statement}

Figure showing samples of $f$ from the prior with Matern kernel at various $L,\nu$ values.
}

\subsection{1c: Solution}
\renewcommand{\thefigure}{1c}
 \begin{figure}[!h]
     \centering
     \includegraphics[width=0.9\textwidth]{example-image-b.pdf}
     \label{fig:1c}
\caption{TODO YOUR BRIEF CAPTION HERE. 
}%endcaption
 \end{figure}

\officialdirections{
\subsection*{1d: Problem Statement}
Provide a short answer (1-3 complete sentences). How does the numerical value of the $\nu$ hyperparameter of the Matern kernel control the qualitative behavior of the sampled function values?
}
\subsection{1d: Solution}

TODO YOUR SOLUTION HERE




\newpage
\officialdirections{
\subsection*{2a: Problem Statement}
Figure showing samples of $f$ from the posterior given the COVID-19 data, with SE kernel at various $L,\tau$ values.
}

\subsection{2a: Solution}
\renewcommand{\thefigure}{2a}
 \begin{figure}[!h]
     \centering
     \includegraphics[width=0.9\textwidth]{example-image-c.pdf}
     \label{fig:2a}
\caption{TODO YOUR BRIEF CAPTION HERE. 
}%endcaption
 \end{figure}
 
\newpage
\officialdirections{
\subsection*{2b: Problem Statement}
Provide a formula and step-by-step explanation for the posterior probability that at a given input value $x_*$, the observed value of $y_*$ is at least $\epsilon$:
You can use results (such as the parametric density family of this distribution and parameter formulas) from lecture/textbook with appropriate citation.
}

\subsection{2b: Solution}

\begin{equation}
p( y_* \geq \epsilon | x_*, y_{1:N}, x_{1:N}) = \text{TODO YOUR MATH HERE}
\end{equation}

\officialdirections{
\subsection*{2c: Problem Statement}
Evaluate your formula above for the scenario where $L=16, \tau=0.05$. At $x_* = +20$, what's the chance that $y_* \geq 4$?
}

\subsection{2c: Solution}

\begin{equation}
p( y_* \geq 4 | x_* = 20, y_{1:N}, x_{1:N}) = ?.???
~~\text{TODO UP TO 3 DECIMAL PLACES}
\end{equation}

\officialdirections{
\subsection*{2d: Problem Statement}
Comment on Near-term Extrapolation. Which of the parameter settings in the 3x3 panel plot from 2a seem to offer the best predictions of COVID-related deaths in the near-term future at $x_* = +20$ (red dashed line). Why?
}

\subsection{2d: Solution}

TODO YOUR ANSWER HERE

\officialdirections{
\subsection*{2e: Problem Statement}
Comment on Far Extrapolation. Set $L=16, \tau=0.05$ (top middle plot in 2a). Report the numerical value of the posterior mean and variance of $f(x_*)$ at $x_* = 365$. Then, make a step-by-step mathematical argument that explains why these values occur, leaning on conceptual intuition rather than just plug-and-chug execution.

}

\subsection{2e: Solution}

TODO YOUR ANSWER HERE


\end{document}
